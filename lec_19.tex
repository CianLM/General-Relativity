\lecture{19}{22/11/2024}{Energy in Gravitational Waves}

\subsection{Energy in Gravitational Waves}

Defining the local energy/local energy flux for the gravitational field is hard in general as one can always choose coordinates such that
\begin{align}
    \partial_\mu g_{\sigma \tau} \bigg|_{p} = 0
.\end{align}

There is no hope of an energy density quadratic in first derivatives.

In the context of perturbation theory, there are various ways to define an energy.

We consider the ungauged vacuum Einstein equations and work to order $\epsilon^2$ such that
\begin{align}
    g_{\mu \nu} = \eta_{\mu \nu} + \epsilon h^{\left( 1 \right) }_{\mu \nu} + \epsilon^2 h^{\left( 2 \right) }_{\mu \nu}
.\end{align}

We observe
\begin{align}
    R_{\mu \nu} \left[ \eta_{\mu \nu} + \epsilon h_{\mu \nu} \right]  = \epsilon R^{\left( 1 \right) }_{\mu \nu}\left[ h \right] + \epsilon^2 R^{\left( 1 \right) }_{\mu \nu} \left[ h \right] 
,\end{align}
where $R^{\left( 1 \right) }_{\mu \nu} \left[ h \right] = \partial^{\rho} \partial_{(\mu} h_{\nu)\rho} - \frac{1}{2} \partial^{\rho} \partial_\rho h_{\mu \nu} - \frac{1}{2} \partial_\mu \partial_\nu h$ are the linear terms and
\begin{align}
    R^{\left( 2 \right) }_{\mu \nu}\left[ h \right] = \frac{1}{2} h^{\rho \sigma} \partial_\mu \partial_\nu h_{\rho \sigma} - h^{\rho \sigma} \partial_\rho \partial_{(\mu} \partial_{\nu) \sigma} + \frac{1}{4} \partial_\mu h_{\rho \sigma} \partial_\nu h^{\rho \sigma} + \partial^{\sigma} \tensor{h}{^{\rho}_\nu} \partial_{[\sigma} h_{\rho] \mu} + \frac{1}{2} \partial_\sigma \left( h^{\sigma \rho} \partial_\rho h_{\mu \nu} \right) \nonumber \\
    - \frac{1}{4} \partial^{\rho} h \partial_\rho h_{\mu \nu} - \left( \partial_\sigma h^{\rho \sigma} - \frac{1}{2} \partial^{\rho} h \right) \partial_{(\mu} h_{\nu)\rho}
,\end{align}
are the quadratic terms. This implies
\begin{align}
    R_{\mu \nu} \left[ \eta_{\mu \nu} + \epsilon h^{\left( 1 \right) }_{\mu \nu} + \epsilon^2 h^{\left( 2 \right) }_{\mu \nu} \right]  = \epsilon R^{\left( 1 \right) }_{\mu \nu}\left[ h^{\left( 1 \right) } \right]  + \epsilon^2 \left( R^{\left( 1 \right) }_{\mu \nu} \left[ h^{\left( 2 \right) } \right] + R^{\left( 2 \right) }_{\mu \nu} \left[ h^{\left( 1 \right) } \right]  \right) 
.\end{align} 

Thus
\begin{align}
    G_{\mu \nu} \left[ \eta + \epsilon h^{\left( 1 \right) } + \epsilon^2 h^{\left( 2 \right) } \right]  = \epsilon G^{\left( 1 \right) }_{\mu \nu} \left[ h^{\left( 1 \right) } \right] + \epsilon^2 \bigg( G^{\left( 1 \right) }_{\mu \nu}\left[ h^{\left( 2 \right) } \right] + \overbrace{R^{\left( 2 \right) }_{\mu \nu} \left[ h^{\left( 1 \right) } \right] - \frac{1}{2} \eta_{\mu \nu} \eta^{\sigma \tau} R^{\left( 2 \right) }_{\sigma \tau} \left[ h^{\left( 1 \right) } \right]}^{-8\pi t_{\mu \nu}\left[ h^{\left( 1 \right) } \right] } \nonumber \\
    + \frac{1}{2} \eta_{\mu \nu} h^{\sigma \tau} R^{\left( 1 \right) }_{\sigma \tau}\left[ h^{\left( 1 \right) } \right] - \frac{1}{2}h_{\mu \nu} \eta^{\sigma \tau} R^{\left( 1 \right) }_{\sigma \tau} \left[ h^{\left( 1 \right) } \right]      \bigg) 
,\end{align}
where
\begin{align}
    G_{\mu \nu}^{\left( 1 \right) }\left[ h \right] = R^{\left( 1 \right) }_{\mu \nu}\left[ h \right] - \frac{1}{2}\eta_{\mu \nu} \eta^{\sigma \tau} R_{\sigma \tau}^{\left( 1 \right) }\left[ h \right] 
.\end{align}

We now consider the contracted Bianchi identity
\begin{align}
    g^{\mu \rho} \nabla_\rho G_{\mu \nu} = 0
,\end{align}
which holds for any metric. Using our expansion of the Einstein tensor we see
\begin{align}\label{eq:contracted_bianchi_expansion}
    0 = \epsilon \eta^{\mu \sigma} \partial_\sigma G^{\left( 1 \right) }_{\mu \nu}\left[ h^{\left( 1 \right) } \right] + \epsilon^2 \left( \eta^{\mu \sigma} \partial_\sigma G^{\left( 1 \right) }_{\mu \nu} \left[ h^{\left( 2 \right) } \right] - 8\pi \partial_\mu \tensor{t}{_{\sigma \nu}} \left[ h^{\left( 1 \right) } \right] \eta^{\mu \sigma} + \sim h^{\left( 1 \right) } \cdot R^{\left( 1 \right) }\left[ h^{\left( 1 \right) } \right]   \right) 
,\end{align}
where the last term is schematic as we can argue it vanishes in a moment.

Considering $G_{\mu \nu}\left[ \eta + \epsilon h^{\left( 1 \right) } + \epsilon^2 h^{\left( 2 \right) } \right] =0$, using our expansion order by order, we deduce
\begin{align}
    G^{\left( 1 \right) }_{\mu \nu}\left[ h^{\left( 1 \right) } \right] = 0 \implies R^{\left( 1 \right) }_{\mu \nu} \left[ h^{\left( 1 \right) } \right] = 0
,\end{align}
and at second order,
\begin{align}
    G^{\left( 1 \right) }_{\mu \nu}\left[ h^{\left( 2 \right) } \right]  = -R^{\left( 2 \right) }_{\mu \nu} \left[ h^{\left( 1 \right) } \right]  + \frac{1}{2} \eta_{\mu \nu} \eta^{\sigma \tau} R^{\left( 2 \right) }_{\sigma \tau} \left[ h^{\left( 1 \right) } \right]  = 8 \pi t_{\mu \nu}\left[ h^{\left( 1 \right) } \right] 
.\end{align}

Thus $h^{\left( 2 \right) }$ solves the linearised Einstein equations sourced by an `\textit{energy momentum tensor}'
\begin{align}
    t_{\mu \nu} = -\frac{1}{8\pi} \left( R^{\left( 2 \right) }_{\mu \nu}\left[ h^{\left( 1 \right) } \right]  - \frac{1}{2} \eta_{\mu \nu}\eta^{\rho \sigma} R^{\left( 2 \right) }_{\rho \sigma} \left[ h^{\left( 1 \right) } \right] \right)   
.\end{align}

From \cref{eq:contracted_bianchi_expansion}, we deduce that
\begin{align}
    \eta^{\mu \sigma} \partial_\mu \tensor{G}{^{\left( 1 \right) }}_{\sigma \nu} \left[ h \right] = 0
,\end{align}
which holds for any perturbation $h$ and $\eta^{\mu \sigma} \partial_\mu t_{\sigma \nu} \left[ h^{\left( 1 \right) } \right] = 0$ when $h^{\left( 1 \right) }$ satisfies the linearised Einstein equations.

We can identify $t_{\mu \nu}$ with the energy momentum of the gravitational field, however it is not gauge invariant. If $h^{\left( 1 \right) }$ decays sufficiently at $\infty$, then
\begin{align}
    \int t_{00} \dd{^3x}
,\end{align}
is invariant and thus the total energy of the field is well defined. However there is no gauge invariant local conservation of energy.

We can get approximate gauge invariance by  \textit{averaging}. Let $W$ be smooth and vanishing for $\left| \vec{x} \right|^2 + t^2 > a$ and satisfy
\begin{align}
    \int_{\R^{4}} W\left( \vec{x},t \right) \dd{^3x} \dd{t} = 1
.\end{align}

We define the average of a tensor in almost inertial coordinates by setting the average to be
\begin{align}
    \left<X_{\mu \nu}\left( x \right)  \right> = \int_{\R^{4}} W \left( y - x \right) X_{\mu \nu}\left( y \right) \dd{^{4}y}
.\end{align}

Suppose we are in the far field regime with radiation of wavelength $\lambda$ and we average over a region of size $a \gg \lambda$. Since $\partial_\mu W \sim  \frac{W}{a}$, we have
\begin{align}
    \left<\partial_\mu X_{\rho \sigma} \right> = \int_{\R^{4}} \partial_\mu W\left( y - x \right) X_{\rho \sigma} \dd{^{4}y} \sim  \frac{\left<X_{\rho \sigma} \right>}{a} \sim  \frac{\lambda}{a} \left<\partial_\mu X_{\rho \sigma} \right>
.\end{align}

We can ignore total derivatives inside averages, and thus
\begin{align}
    \left<A \partial B \right> \left<\partial \left( AB \right)  \right> - \left<\left( \partial A \right) B \right> \sim - \left<\left( \partial A \right) B \right>
.\end{align}

With this we can show
\begin{exercise}
    If $h$ solves the vacuum linearised Einstein equations,
    \begin{align}
        \left<\eta^{\mu \nu} R^{\left( 2 \right) }_{\mu \nu}\left[ h \right]  \right> =0
    ,\end{align}
    and 
    \begin{align}
        \left<t_{\mu \nu} \right> = \frac{1}{32\pi} \left<\partial_\mu \overline{h}_{\rho \sigma} \partial_\nu \overline{h}^{\rho \sigma} -\frac{1}{2} \partial_\mu \overline{h} \partial_\nu \overline{h} - 2 \partial_\sigma \overline{h}^{\rho \sigma} \partial_{(\mu} \overline{h}_{\nu)\rho}\right> 
    ,\end{align}
    where we raise and lower with $\eta$. Lastly, $\left<t_{\mu \nu} \right>$ is gauge invariant.
\end{exercise}

Using this formula, and the results of last lectures, we can find the energy lost by a system producing gravitational waves.

The average spatial energy flux is $S_i = - \left<t_{0i} \right>$. We calculate the average energy flux across a sphere of radius $r$ centered on the source. This is given by
\begin{align}
    \left<P \right> = - \int_{S_r} r^2 \dd{\Omega} \left<t_{0i} \right> \hat{x}_i 
.\end{align}

In wave gauge, we have
\begin{align}
    \left<t_{0i} \right> &= \frac{1}{32\pi} \left<\partial_0 \overline{h}_{\rho \sigma} \partial_i \overline{h}^{\rho \sigma} - \frac{1}{2} \partial_0 \overline{h} \partial_i \overline{h} \right> \\
    &= \frac{1}{32\pi} \left<\partial_0 \overline{h}_{jk} \partial_i \overline{h}_{jk} - 2 \partial_0 \overline{h}_{0j} \partial_i \overline{h}_{0j} + \partial_0 \overline{h}_{00} \partial_i \overline{h}_{00} - \frac{1}{2} \partial_0 \overline{h} \partial_i \overline{h} \right> 
.\end{align}

Using $\overline{h}_{ij} =\frac{2}{r} \ddot{I}_{ij}\left( t - r \right) $,
\begin{align}
    \partial_i \overline{h}_{jk} = \frac{2}{r} \dddot{I}_{jk} \left( t - r \right) 
,\end{align}
and
\begin{align}
    \partial_i \overline{h}_{jk} = \left( -\frac{2}{r} \dddot{I}_{jk}\left( t - r \right) - \frac{2}{r^2} \underbrace{\ddot{I}_{jk}\left( t - r \right)}_{\text{0 in rad. gauge}}   \right) \hat{x}_i
.\end{align}

Therefore
\begin{align}
    -\frac{1}{32\pi} \int r^2 \dd{\Omega} \left<\partial_0 \overline{h}_{jk} \partial_i \overline{h}_{jk} \right> \hat{x}_i = \frac{1}{2} \left<\dot{\ddot{I}}_{ij} \dot{\ddot{I}}_{ij} \right>_{t - r}
,\end{align}
where this is an average over a window centered at $t - r$.
