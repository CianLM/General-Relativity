\lecture{15}{12/11/2024}{Gauges in Gravity}

There are no nontrivial $\chi$ such that $\widetilde{A}_\nu$ also satisfies $\partial_\mu \widetilde{A} =0$ and satisfies the same initial conditions.

If we solve the wave equation, we see that it is sufficient to set
\begin{align}
    \partial_\mu A^{\mu} \bigg|_{t = 0} = 0 && \partial_0 \left( \partial_\mu A^{\mu} \right) \bigg|_{t = 0} = 0
.\end{align}

Then we find that $\partial_\mu A^{\mu} = 0$ and thus our solution solves the wave and hence Maxwell's equation.


\subsection{Gauge Freedom in General Relativity}


If $\left( M,g \right) $ solves Einstein's equation with energy momentum tensor $T$ and $\phi : M \to M$ a diffeomorphism, then $\phi^{*} g$ also solves the Einstein equations with energy momentum tensor $\phi^{*} T$. At a local level, this is the coordinate independence that we have built into general relativity.

This gives us an infinite dimensional set of solutions identical to the gauge freedom we saw for Maxwell. There are several approaches to fixing the coordinates. 

We consider the harmonic gauge.

\begin{lemma}
    In any local coordinate system
    \begin{align}
        R_{\rho \delta \mu \nu} = \frac{1}{2} \left( g_{\rho \nu , \mu \delta} + g_{\delta \mu , \nu \rho} - g_{\rho \mu, \nu \delta} - g_{\delta \nu, \mu \rho} \right) - \tensor{\Gamma}{_{\mu \lambda \rho}} \tensor{\Gamma}{_\nu^{\lambda}_{\delta}} + \tensor{\Gamma}{_{\nu \lambda \rho}} \tensor{\Gamma}{_{\mu}^{\lambda}_\delta}
    ,\end{align}
    and 
    \begin{align}
        R_{\delta \nu} &= -\frac{1}{2} \overbrace{g^{\mu \rho} g_{\delta \nu , \mu \rho}}^{g^{\mu \rho} \partial_\mu \partial_\rho  g_{\delta \nu}} + \overbrace{\frac{1}{2} \partial_\delta \tensor{\Gamma}{_{\mu \nu}^{\mu}} + \frac{1}{2} \partial_\nu \tensor{\Gamma}{_{\mu \sigma}^{\mu}} - \tensor{\Gamma}{_{\mu \lambda}^{\mu}} \tensor{\Gamma}{_\nu^{\lambda}_\delta}}^{0} \nonumber\\
        &\quad \, + \tensor{\Gamma}{_{\lambda \tau \nu}} \tensor{\Gamma}{^{\lambda \tau}_{\delta}} + \tensor{\Gamma}{_{\lambda \tau \nu}} \tensor{\Gamma}{^{\tau}_\sigma^{\lambda}} + \tensor{\Gamma}{_{\lambda \tau \sigma}} \tensor{\Gamma}{^{\tau}_{\nu}^{\lambda}} 
    .\end{align}
\end{lemma}

\begin{proof}
    
\end{proof}

Where notice that if $g = \eta$, then $g^{\mu \rho} \partial_\mu \partial_\rho  g_{\delta \nu}$ is the wave operator. As we are now in a globally hyperbolic setting (i.e. Lorentzian), this is a curved wave operator. This is a quasi-linear second order ODE.

This form of Ricci is well-adapted to wave/harmonic coordinates. Suppose we choose a system of coordinates $\{x^{\nu}\}$ which satisfy the wave equation, $\nabla_\mu \nabla^{\mu} x^{\nu} = 0$ where $\nu$ here is a label not a vector component. We then see that
\begin{align}
    0 &= \nabla^{\mu} \partial_\mu x^{\nu} \\
    &= \partial^{\mu} \partial_\mu x^{\nu} + \tensor{\Gamma}{^{\mu \delta}_\mu} \partial_\delta x^{\nu} \\
    0 &= \tensor{\Gamma}{^{\mu \nu}_{\mu}}
.\end{align}

Writing the Christoffel symbol in terms of the metric, we see this implies
\begin{align} \label{eq:wave_gauge}
    \frac{1}{2} g^{\mu \delta} \left( g_{\mu \kappa ,\sigma} - \frac{1}{2} g_{\mu \sigma , \kappa} \right)  = 0
.\end{align}

For such coordinates, $R_{\delta \nu} = $
\begin{align}
            R_{\delta \nu} &= -\frac{1}{2} g^{\mu \rho} g_{\delta \nu , \mu \rho}  + \tensor{\Gamma}{_{\lambda \tau \nu}} \tensor{\Gamma}{^{\lambda \tau}_{\delta}} + \tensor{\Gamma}{_{\lambda \tau \nu}} \tensor{\Gamma}{^{\tau}_\sigma^{\lambda}} + \tensor{\Gamma}{_{\lambda \tau \sigma}} \tensor{\Gamma}{^{\tau}_{\nu}^{\lambda}} 
.\end{align}

These Christoffel symbol terms are quadratic in the metric $g$ and its derivative $\partial g$. Therefore $R_{\sigma \nu} = 0$ reduces to a system of nonlinear wave equations in this gauge. We \emph{can} solve this (locally in time) given initial data. 

Further, we can show that if the gauge condition is initially satisfied, it will remain true for all $\left( t,x \right) $ where the solution is defined. This was proved by Choquet-Bruhart in 1954.

\subsection{Linearised theory}

Suppose we are in a situation where the gravitational field is weak. We expect to be able to describe the metric as a perturbation of Minkowski such that
\begin{align}
    g_{\mu \nu} = \eta_{\mu \nu} + \epsilon h_{\mu \nu}
,\end{align}
where $\eta_{\mu \nu} = \text{diag}\left( -1,1,1,1 \right)$ and $\epsilon \ll 1$ is a small parameter such that we neglect terms of $\mathcal{O}\left( \epsilon^2 \right)$. 

If the metric has this form we say we're in ``almost inertial'' coordinates. One can check $g^{\mu \nu} = \eta^{\mu \nu} - \epsilon h^{\mu \nu}$ where $h^{\mu \nu} = \eta^{\mu \delta} h_{\delta \tau} \eta^{\tau \nu}$. Namely, for $\mathcal{O}\left( \epsilon \right) $, quantities, one can raise/lower with $\eta$.

Suppose that our metric is also in wave gauge, we then have \cref{eq:wave_gauge} and then
\begin{align}
    0 = g^{\mu \delta} \left( g_{\mu \kappa , \delta} - \frac{1}{2} g_{\mu \delta , \kappa} \right) = \epsilon \partial^{\mu} \left( h_{\mu \kappa} - \frac{1}{2} \eta_{\mu \kappa} h \right) 
,\end{align}
where $h = \eta^{\mu \nu} h_{\mu \nu}$. Then using our expression for the Ricci tensor,
\begin{align}
    R_{\mu \nu} = -\frac{\epsilon}{2} \eta^{\delta \tau} \partial_\delta \partial_\tau h_{\mu \nu}
.\end{align}

In order for the Einstein equation to hold, we must have the stress energy tensor $\mathbb{T}_{\mu \nu} = \epsilon T_{\mu \nu}$.

Then to order $\epsilon$ in the Einstein equation, we have
\begin{align}
    -\frac{1}{2} \Box h_{\mu \nu} + \frac{1}{4} \eta_{\mu \nu} \eta^{\delta \tau} \partial_\delta \partial_\tau h &= 8 \pi G T_{\mu \nu} 
\end{align}

    \begin{align}
        \fbox{$\displaystyle \Box \overline{h}_{\mu \nu} = -16 \pi G T_{\mu \nu},$} &&
        \fbox{$\displaystyle \partial_\mu \tensor{\overline{h}}{^{\mu}_\nu} = 0,$}
\end{align}
where $\overline{h}_{\mu \nu} = h_{\mu \nu} -\frac{1}{2} h \eta_{\mu \nu}$ is the \textit{trace reversed metric perturbation}.

These are the linearised Einstein equation in wave (also harmonic, Lorentz, de Donder) gauge. We can solve these given initial data at $t = 0$.

If the data satisfies the gauge condition, $\partial_\mu \tensor{\overline{h}}{^{\mu}_{\nu}} \bigg|_{t=0} = \partial_0 \left( \partial_\mu \tensor{\overline{h}}{^{\mu}_\nu} \right) \bigg|_{t=0} =0$, then we can check that this gauge condition is propagated. 

Since $\partial_\mu \tensor{T}{^{\mu}_\nu} = 0$, we have $\Box \left( \partial_\mu \tensor{\overline{h}}{^{\mu}_{\nu}} \right) = 0$, so the gauge condition holds for all times.

\subsection{Linearised gauges}

Suppose we have a physically equivalent solution that is not necessarily in wave gauge. We've seen that 2 such equivalent solutions are related by a diffeomorphism. In order that the diffeomorphism reduces smoothly to the identity as $\epsilon \to 0$, it should take the form $\phi_{\epsilon}^{\xi}$ for some vector field $\xi$. If $S$ is any tensor field, then from the definition of the Lie derivative,
\begin{align}
    \left( \Phi_{\epsilon}^{\xi} \right) S = S + \epsilon \mathcal{L}_{\xi} S + \mathcal{O}\left( \epsilon^2 \right) 
.\end{align}

In particular,
\begin{align}
    \left( \left( \Phi_{\epsilon}^{\xi} \right)^{*} \eta \right)_{\mu \nu} = \eta_{\mu \nu} + \epsilon \underbrace{\left( \partial_\mu \xi_\nu + \partial_\nu \xi_\mu \right)}_{h_{\mu \nu}} + \mathcal{O}\left( \epsilon^2 \right)  
.\end{align}

Solutions that take this form are called a \emph{linear gauge transformation}.

