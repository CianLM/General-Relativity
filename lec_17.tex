\lecture{17}{18/11/2024}{Gravitational Waves}

We seek a propagating wave solution the vacuum linearised Einstein equations in wave gauge,
\begin{align}
    \partial_\mu \partial^{\mu} \overline{h}_{\nu \rho} = 0 && \partial_\mu \tensor{\overline{h}}{^{\mu}_{\nu}}
.\end{align}

Making the ansatz,
\begin{align}
    \overline{h}_{\mu \nu} = \Re \left( H_{\mu \nu} e^{-i k^{\mu} x_{\mu}} \right) 
,\end{align}
with $H_{\mu \nu} = H_{\mu \nu}$ and $k^{\mu}$ both constant.

We see this is a solution if
\begin{align}
    k^{\mu} k_\mu = 0 && k_{\mu} \tensor{H}{^{\mu}_\nu} = 0
,\end{align}
which implies $k^{\mu}$ is null and thus waves travel at the speed of light. Take $k^{\mu} = \omega \left( 1,0,0,1 \right) $ without loss of generality.

Note that there is still some residual gauge freedom: recalling that a diffeomorphism generated by $\xi$ acts on $\partial^{\mu} \overline{h}_{\mu \nu}$ by
\begin{align}
    \partial^{\mu} \overline{h}_{\mu \nu} \to \partial^{\mu} \overline{h}_{\mu \nu} + \partial_\mu \partial^{\mu} \xi_\nu
.\end{align}

If $\partial_\mu \partial^{\mu} \xi_\nu = 0$, then the gauge condition is preserved. 

Let
\begin{align}
    \xi_\nu = \text{Re}\left( -i X_\nu e^{i k^{\mu} x_\mu} \right) 
.\end{align}

Then $\partial_\mu \partial^{\mu} \xi_\nu = 0$, and one can check
\begin{align}
    H_{\mu \nu} \to H_{\mu \nu} + k_\mu X_\nu + k_\nu X_\mu - \eta_{\mu \nu} k_\sigma X^{\sigma}
.\end{align}

\begin{exercise}
    Show that if we take $X_0 = 0$ and $X_{i} = - \frac{H_{0i}}{k_0}$, then we can set $H_{0\mu} = 0$.

    Making a further transformation of the form
    \begin{align}
        X_0 = \alpha k_0 && X_{i} = - \alpha k_{i}
    ,\end{align}
    show we can additionally impose $\tensor{H}{_{\mu}^{\mu}} = 0$, which implies $h_{\mu \nu} = \overline{h}_{\mu \nu}$. 

    These conditions together are called \textit{transverse traceless}.
\end{exercise}

Since $H_{0 \mu}$ and $H_{\nu \mu} k^{\nu} = \omega \left( H_{0 \mu} + H_{3\mu} \right) = 0$, we deduce $H_{3\mu} = 0$. With symmetry and tracelessness of $H_{\mu \nu}$, we have
\begin{align}
    H_{\mu \nu} = \mqty( 0 & 0 & 0& 0 \\
    0 & h_+ & H_{\times } & 0 \\ 
    0 & H_{\times } & -H_+ & 0 \\
    0 & 0 & 0 & 0 )
,\end{align}
where $H_{+}$ and $H_{\times }$ are constants corresponding to two independent polarizations of the wave.

To understand the consequences of such a gravitational wave, we recall the geodesic deviation equation. If $\lambda$ is a geodesic with tangent vector $T$, then a vector $Y$ joining $\lambda$ to a nearby geodesic satisfies $T^{a} \nabla_a \left( T^{b} \nabla_b Y^{c} \right) = \tensor{R}{^{c}_{dab}} T^{d} T^{a} Y^{b}$.

Suppose a freely falling observer sets up a frame consisting of $e_0 = T$ (tangent to the observer's worldline), together with three spacelike vectors $e_1, e_2$ and $e_3$ which are parallel transported along the world line and initially satisfy $g\left( e_\mu, e_\nu \right) = \eta_{\mu \nu}$. Since parallel transport preserves inner products, we can assume $\{e_{\mu}\} $ is orthonormal at all times.

Since $T^{a} \nabla_a \left( e_\alpha \right) = 0$, the geodesic equation implies
\begin{align}
    T^{a} \nabla_a \left( T^{b} \nabla_b \left( Y_c \tensor{e}{^{c}_\alpha} \right)  \right) = \tensor{R}{_{abcd}} \tensor{e}{^{a}_\alpha} \tensor{e}{^{b}_{0}} \tensor{e}{^{c}_{0}} Y^{d}
.\end{align}

Now, $Y_c \tensor{e}{^{c}_\alpha} \equiv Y_{\alpha}$ is a scalar, so the equation becomes 
\begin{align}
    \dv[2]{Y_\alpha}{\tau} = R_{abcd} \tensor{e}{^{a}_\alpha} \tensor{e}{^{b}_0} \tensor{e}{^{c}_0} \tensor{Y}{^{\beta}} \tensor{e}{^{d}_\beta}
.\end{align}

For our problem of a gravitational wave spacetime, the Riemann curvature is $\mathcal{O}\left( \epsilon \right) $, so we only need $e_\alpha$ to leading order. We can assume $e_0 = \partial_t$, $e_i = \partial_i$ and $\lambda$ is $\tau \mapsto \left( \tau,0,0,0 \right) $.

Then
\begin{align}
    \dv[2]{Y_\alpha}{\tau} = R_{\alpha 0 0 \beta} Y^{\beta} + \mathcal{O}\left( \epsilon^2 \right) 
.\end{align}

To order $\epsilon$, we have
\begin{align}
    R_{\rho \sigma \mu \nu} = \frac{1}{2} \left( h_{\rho \nu , \mu \sigma} + h_{\sigma \mu , \nu \rho} - h_{\rho \mu , \nu \sigma} - h_{\sigma \nu , \mu \rho} \right) 
,\end{align}
but since $h_{0 \mu} = 0$, we find
\begin{align}
    R_{\alpha 0 0 \beta} = \frac{1}{2} h_{\alpha \beta , 0 0}
.\end{align}

So
\begin{align}
    \dv[2]{Y_\alpha}{t} = \frac{1}{2} \pdv[2]{h_{\alpha \beta}}{t} Y^{\beta}
,\end{align}
where we used $t = \tau$ to order $\epsilon$.

Let's consider the $+$ polarization, where
\begin{align}
    h_{\mu \nu} = \Re \left( H_{\mu \nu} e^{i k^{\mu} x_{\mu}} \right) = H_+ \mqty( 0 & 0 & 0 & 0 \\ 0& 1 & 0 & 0 \\ 0 & 0 & -1 & 0 \\ 0 & 0 & 0& 0 ) \cos \left( \omega \left( t- z \right)  \right) 
.\end{align}

As we are interested in this along the world line $\lambda$, we can set $z = 0$. Then
\begin{align}
    \dv[2]{Y^{0}}{t} = \dv[2]{Y^{3}}{t} = 0
,\end{align}
and
\begin{align}
    \dv[2]{Y^{1}}{t} &= -\frac{1}{2}\omega^2 \left| H_+ \right| \cos \left( \omega \left( t- t_0 \right)  \right) Y^{1} \\
    \dv[2]{Y^{2}}{t} &= \frac{1}{2} \omega^2 \left| H_+ \right| \cos \left( \omega \left( t - t_0 \right)  \right) Y^2
.\end{align}

As $H_+$ is small, we solve perturbatively with $\dv{Y^{1}}{t} = \dv{Y^{2}}{t} = 0$ finding
\begin{align}
    Y^{1} &= Y^{1}_0 \left( 1 + \frac{1}{2} \left| H_+ \right| \cos \left( \omega \left( t -t_0 \right)  \right)  \right) \\
    Y^{2} &= Y^{2}_0 \left( 1 - \frac{1}{2} \left| H_+ \right| \cos \left( \omega \left( t -t_0 \right)  \right)  \right)
.\end{align}

This is a periodic ellipse oscillating.

\begin{exercise}
    Find the solution for the cross polarization.
\end{exercise}
