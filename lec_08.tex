\lecture{8}{28/10/2024}{Christoffel Symbols}

\subsection{Geodesic Equation}

Therefore we fix the parameterisation such that the curve is parameterized by the proper time $\tau$ itself. Doing this, since
\begin{align}
    \dv{x^{\mu}}{\tau} = \dot{x}^{\mu} \dv{u}{\tau} \text{~and~} -1 = g_{\mu \nu} \dv{x^{\mu}}{\tau} \dv{x^{\nu}}{\tau}
,\end{align}
we have that
\begin{align}
    -1 = g_{\mu \nu} \dot{x}^{\mu} \dot{x}^{\nu} \left( \dv{u}{\tau} \right)^2 \implies \dv{u}{\tau} =  \frac{1}{\sqrt{G}}
,\end{align}
which then implies
\begin{align}
    \frac{1}{G} \dv{u} = \dv{\tau}
.\end{align}

Returning to the Euler Lagrange equation, we find that we can write it as
\begin{align}
    \dv{\tau} \left( g_{\mu \nu} \dv{x^{\nu}}{\tau} \right)  = \frac{1}{2} g_{\nu \rho , \mu} \dv{x^{\nu}}{\tau} \dv{x^{\rho}}{\tau}
.\end{align}

% fill in

This then becomes
\begin{align}
    g_{\mu \nu} \dv[2]{x^{\nu}}{\tau} + g_{\mu \nu , \rho} \dv{x^{\nu}}{\tau} \dv{x^{\rho}}{\tau} - \frac{1}{2} g_{\sigma \rho, \mu} \dv{x^{\sigma}}{\tau} \dv{x^{\rho}}{\tau} = 0
.\end{align}

Where notice that we can replace
\begin{align}
    g_{\mu \nu, \rho} \dv{x^{\nu}}{\tau} \dv{x^{\rho}}{\tau} = g_{\mu ( \nu , \rho )} \dv{x^{\nu}}{\tau} \dv{x^{\rho}}{\tau}
,\end{align}
as it is symmetric in $\nu$ and $\rho$.

Thus, notice that we can write this as
\begin{align} \label{eq:geodesic}
    \dv[2]{x^{\nu}}{\tau} + \tensor{\Gamma}{_{\mu}^{\nu}_\rho} \dv{x^{\mu}}{\tau} \dv{x^{\rho}}{\tau} = 0
,\end{align}
where 
\begin{align}
    \tensor{\Gamma}{_{\mu}^{\nu}_\rho} \equiv \frac{1}{2} g^{\nu \sigma} \left( g_{\mu \sigma , \rho} + g_{\sigma \rho , \mu} - g_{\nu \rho , \sigma} \right) 
,\end{align}
are the \textbf{Christoffel symbols} of $g$.

\begin{notes}~
    \begin{itemize}
        \item These symbols have a symmetry such that
        \begin{align}
            \tensor{\Gamma}{_{\nu}^{\mu}_\rho} = \tensor{\Gamma}{_{\rho}^{\mu}_\nu}
        .\end{align}
        
        \item Christoffel symbols are \textbf{not} tensor components as they do not transform desirably under coordinate transformations.
        
        \item Solutions to \cref{eq:geodesic} are obtainable with standard ODE theory. Such solutions are called \textbf{geodesics}.
        
        \item The same equation governs curves of extremal length in a Riemannian manifold (or spacelike curves in a Lorentzian manifold) parameterized by arc-length.
    \end{itemize}
\end{notes}

\begin{exercise}
    Show that \cref{eq:geodesic} can be obtained as the Euler-Lagrange equation for the Lagrangian
    \begin{align}
        L = - g_{\mu \nu} \left( x \left( \tau \right)  \right) \dot{x}^{\mu} \left( \tau \right)  \dot{x}^{\nu} \left( \tau \right) 
    .\end{align}
\end{exercise}

\begin{examples}~
    \begin{enumerate}[label=\arabic*)]
        \item In Minkowski space, in an inertial frame $g_{\mu \nu} = \eta_{\mu \nu}$ so $\tensor{\Gamma}{_{\mu}^{\nu}_\rho} = 0$ and the geodesic equation is
            \begin{align}
                \dv[2]{x^{\mu}}{\tau} = 0
            ,\end{align}
            which has geodesics (solutions) which are straight lines.
        \item The Schwarzschild metric in Schwarzschild coordinates is a metric on $M = \R_t \times \left( 2m, \infty \right)_r \times S^2_{\theta, \phi} $ given by
            \begin{align}
                \dd{s}^2 = - f \dd{t}^2 + \frac{\dd{r}^2}{f} + r^2 \left( \dd{\theta}^2 + \sin^2 \theta \dd{\phi}^2 \right) 
            ,\end{align}
            where $f = 1 - \frac{2m}{r}$.

            One can then write the Lagrangian as
            \begin{align}
                L = f \left( \dv{t}{\tau} \right)^2 - \frac{1}{f} \left( \dv{r}{\tau} \right)^2 - r^2 \left( \dv{\theta}{\tau} \right)^2 - r^2 \sin^2 \theta \left( \pdv{\phi}{t} \right)^2
            .\end{align}

            The Euler-Lagrange equation for $t \left( \tau \right) $ is 
            \begin{align}
                \dv{\tau} \left( \pdv{L}{t'} \right) = \pdv{L}{t}
            ,\end{align}
            where $t' = \dv{t}{\tau}$. This gives us
            \begin{align}
                2 \dv{\tau} \left( f \dv{t}{\tau} \right) = 0
            ,\end{align}
            which implies
            \begin{align}
                f \dv[2]{t}{\tau} + \dv{f}{r} \left( \dv{r}{\tau} \right) \left( \dv{t}{\tau} \right)  &= 0
            .\end{align}

            Comparing this to the geodesic equation \cref{eq:geodesic}, we see
            \begin{align}
                \tensor{\Gamma}{_1^{0}_0} = \tensor{\Gamma}{_0^{0}_1} = \frac{1}{2} \frac{1}{f} \dv{f}{r}
            ,\end{align}
            and $\tensor{\Gamma}{_\mu^{0}_\nu} = 0$ otherwise. The rest of the symbols can be found from the other Euler Lagrange equations.
    \end{enumerate}
\end{examples}

\subsection{Covariant Derivative}

For a function $f : M \to \R$, we know that
\begin{align}
    \pdv{f}{x^{\mu}}  \text{~are the components of a covector $\left( df \right)_a$}
.\end{align}

For a vector field we can't just differentiate it's components as the basis vectors themselves can have spatial dependence.

\begin{exercise}
    Show that if $V$ is a vector field, then
    \begin{align}
        \tensor{T}{^{\mu}_\nu} := \pdv{V^{\mu}}{x^{\nu}}
    ,\end{align}
    are not the components of a $\left( 1,1 \right) $ tensor.
\end{exercise}

\begin{definition}
    A \textbf{covariant derivative} $\nabla$ on a manifold $M$ is a map sending smooth vector fields $X$, $Y$ to a vector field $\nabla_X Y$ satisfying
    \begin{enumerate}[label=\roman*)]
        \item linearity in the first vector such that
            \begin{align}
        \nabla_{f X + g Y} Z = f \nabla_X Z + g \nabla_Y Z
        ,\end{align}
        \item linearity in the second such that
        \begin{align}
            \nabla_X \left( Y + Z \right) = \nabla_X Y + \nabla_X Z
        ,\end{align}
        \item and a Leibniz rule such that
        \begin{align}
            \nabla_X \left( f Y \right) = f \nabla_X Y + \left( \nabla_X f \right) Y
        ,\end{align}
        where $\nabla_X f = X \left( f \right) $ and $X,Y,Z$ are smooth vector fields and $f,g$ are functions.
\end{enumerate}
\end{definition}

\begin{note}
    This first condition implies that $\nabla Y : X \mapsto \nabla_X Y$ is a linear map of $T_{p} M$ to itself and so defines a $\left( 1,1 \right) $ tensor, which we call the covariant derivative of $Y$.
\end{note}

In abstract index notation, one can write
\begin{align}
    \tensor{\left( \nabla Y \right)}{^{a}_b} = \nabla_b Y^{a}  \text{~or~} \tensor{Y}{^{a}_{;b}}
.\end{align}

\begin{definition}
    In a basis $\{e_{\mu}\} $ the \textbf{connection components} $\tensor{\Gamma}{_{\nu}^{\mu}_{\rho}}$ are defined by
    \begin{align}
        \nabla_{e_{\rho}} e_{\nu} = \tensor{\Gamma}{_{\nu}^{\mu}_\rho} e_{\mu}
    .\end{align}
\end{definition}

Once we know these connection components, they completely determine $\nabla$. Namely, take
\begin{align}
    \nabla_{X} Y  &= \nabla_{X^{\mu}e_{\mu}} \left( Y^{\nu} e_{\nu} \right)  \\
                  &\overset{\text{i)}}{=} X^{\mu} \nabla_{e_{\mu}} \left( Y^{\nu} e_{\nu} \right)  \\
                  &\overset{\text{ii) \& iii)}}{=} X^{\mu}  \left( e_{\mu} \left( Y^{\nu} \right) e_{\nu} + Y^{\sigma} \nabla_{e_\mu} e_\sigma \right)  \\
                  &= \left( X^{\mu} e_\mu \left( Y^{\nu} \right) + \tensor{\Gamma}{_{\sigma}^{\nu}_{\mu}} Y^{\sigma} X^{\mu}  \right) e_{\nu}
.\end{align}

Hence the components of the covariant derivative can be written as
\begin{align}
    \left( \nabla_X Y \right)^{\nu} = X^{\mu} \left( e_\mu \left( Y^{\nu} \right) + \tensor{\Gamma}{_{\sigma}^{\nu}_\mu} Y^{\sigma} \right) 
,\end{align}
or identically, in different notation,
\begin{align}
    \tensor{Y}{^{\nu}_{; \mu}} &= e_{\mu} \left( Y^{\nu} \right)  + \tensor{\Gamma}{_{\sigma}^{\nu}_{\mu}} Y^{\sigma} \\
    &= \tensor{Y}{^{\nu}_{,\mu}} + \tensor{\Gamma}{_{\sigma}^{\nu}_{\mu}} Y^{\sigma}
,\end{align}
where $\tensor{Y}{^{\nu}_{,\mu}} = \pdv{Y^{\nu}}{x^{\mu}}$.

\begin{note}
    Remember that $\tensor{\Gamma}{_{\mu}^{\nu}_\sigma}$ are not the components of a tensor, hence we call them \textit{symbols}, like the Levi-Civita symbol $\epsilon_{\mu \nu \rho \tau}$.
\end{note}

We extend $\nabla$ to arbitrary tensor field by requiring the Leibniz property holds.
\begin{example}
    For a tensor field $\eta$, we define
    \begin{align}
        \left( \nabla_{X} \eta \right)  \left( Y \right) := \nabla_X \left( \eta \left( Y \right)  \right) - \eta \left( \nabla_X Y \right) 
    .\end{align}
    In component form, we can write this as
    \begin{align}
        \left( \nabla_X \eta \right) Y &= X^{\mu} e_\mu \left( \eta_\sigma Y^{\sigma} \right) - \eta_{\sigma} \left( \nabla_{X} Y \right)^{\sigma} \\
        &= X^{\mu} e_\mu \left( \eta_\sigma \right) Y^{\sigma} + X^{\mu} \eta_\sigma e_\mu \left( Y^{\sigma}\right) - \eta_\sigma \left( X^{\nu} e_\nu \left( Y^{\sigma}   \right) + X^{\nu} \tensor{\Gamma}{_{\tau}^{\sigma}_{\nu}} Y^{\tau} \right)  \\
        &= \left( e_{\mu} \left( \eta_{\sigma} \right) - \tensor{\Gamma}{_{\sigma}^{\nu}_{\mu}} \eta_{\nu} \right) X^{\mu} Y^{\sigma}
    ,\end{align}
    and thus as $\grad \eta$ is linear in both $X$ and $Y$, it is a (0,2) tensor (it also transforms appropriately).
    Therefore, with respect to our basis, we have
    \begin{align}
        \nabla_\mu \eta_\sigma = e_\mu \left( \eta_\sigma \right)  - \tensor{\Gamma}{_{\sigma}^{\nu}_\mu} \eta_\nu =: \eta_{\sigma ; \mu} \\
        \implies \eta_{\sigma ; \mu} = \eta_{\sigma , \mu} - \tensor{\Gamma}{_{\sigma}^{\nu}_\mu} \eta_\nu
    .\end{align}
\end{example}







