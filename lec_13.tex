\lecture{13}{8/11/2024}{Lie Derivatives}

\begin{exercise}
    Let $\left( N,g \right) = \left( \R^3, \delta \right) $, $M = S^2$. Let $\phi$ be the map taking a point on $S^2 $ with spherical coordinates $\left( \theta, \phi \right) $ to 
    \begin{align}
        \left( x^{1}, x^{2}, x^{3} \right) = \left( \cos \theta \cos \phi, \sin \theta \sin \phi, \cos \theta \right) 
    ,\end{align}
    where
    \begin{align}
        \phi^{*}\left( \left( \dd{x}^{1} \right)^2 + \left( \dd{x}^{2} \right)^2 + \left( \dd{x}^{3} \right)^2 \right) = \dd{\theta}^2 + \sin^2\theta \dd{\phi}^2
    .\end{align}
\end{exercise}

\begin{proof}
    
\end{proof}

If $\phi$ is an immersion and $\left( N,g \right) $ is \textit{Lorentzian}, then $\phi^* g$ is \textit{not} in general a metric on $M$. There are 3 important cases
\begin{itemize}
    \item If $\phi^{*} g$ is a Riemannian metric, we say $\phi \left( M \right) $ is \textit{spacelike}.
    \item If $\phi^{*} g$ is a Lorentzian metric, we say $\phi \left( M \right) $ is \textit{timelike}.
    \item If $\phi^{*} g$ is everywhere degenerate, we say $\phi \left( M \right) $ is \textit{null}.
\end{itemize}

Recall that $\phi : M \to N$ is a diffeomorphism if it is bijective with a smooth inverse. If we have a diffeomorphism, we can push forward a general $\left( r,s \right) $ tensor at $p$ to an $\left( r,s \right) $ tensor at $\phi \left( p \right) $ by
\begin{align}
    \phi_{*} T \left( \eta^{1}, \cdots, \eta^{r}, X_1 \cdots , X_s \right) = T \left( \phi^{*} \eta^{1}, \cdots, \phi^{*} \eta^{r}, \phi_*^{-1} X_1,\cdots, \phi_*^{-1} X_s \right) 
,\end{align}
$\forall  \eta_i \in T_{\phi \left( p \right) }^{*} N$ and $X_i \in T_{\phi \left( p \right) } N$. Namely, we define a pullback by $\phi_*^{-1} = \phi^{*}$.

If $M, N$ are diffeomorphic, we often don't distinguish between them. We can think of $\phi : M \to M$ as a symmetry of $T$ if $\phi_* T = T$. If $T$ is the metric, we say $\phi$ is an \textbf{isometry}.

\begin{example}
    In Minkowski space, with an inertial frame,
    \begin{align}
        \phi \left( x^{0}, x^{1},\cdots,x^{n} \right) = \left( x^{0} + 1, x^{1}, \cdots, x^{n} \right) 
    ,\end{align}
    is a symmetry of $g$.
\end{example}

An important class of diffeomorphisms are those generated by a vector field. If $X$ is a smooth vector field, we associate to each point $p \in M$ the point $\phi^{X}_t \left( p \right) \in M$ given by flowing a parameter distance $t$ along the integral curve of $X$ starting at $p$.

Suppose $\phi_t^{X}\left( p \right) $ is well defined for all $t \in I \subset \R$ for each $p \in M$. Then $\phi_t^{X} : M \to M$ is a diffeomorphism for all $t \in I$.

Further, we have some nice properties.
\begin{itemize}
    \item If $t,s,t+s \in I$, Then $\phi_t^{X} \circ \phi_s^{X} = \phi_{t + s}^{X}$ and $\phi_0^{X} = \text{id}$. If $I = \R$, this gives $\{\phi_t^{X}\}_{t \in \R}$ the structure of a one parameter abelian group.
    \item If $\phi_t$ is any smooth family of diffeomorphisms satisfying these group conditions, we can define a vector field by $X_p = \dv{t} \left( \phi_t \left( p \right)  \right) \bigg|_{t=0}$. Then $\phi_t = \phi_t^{X}$.
\end{itemize}


\subsection{The Lie Derivative}

We can use $\phi_t^{X}$ to compare tensors at different points. As $t \to 0$, this gives a new notion of a derivative, the \textbf{Lie derivative}.

Suppose $\phi_t^{X} : M \to M$ is the smooth one parameter family of diffeomorphisms generated by the vector field $X$.

\begin{definition}
    For a tensor field $T$, the \textbf{Lie derivative} of $T$ with respect to $X$ is
    \begin{align}
        \left( \mathcal{L}_X T \right)_p = \lim_{t \to 0} \frac{\left( \left( \phi_t^{X} \right)^{*} T \right)_p - T_p}{t}
    .\end{align}
\end{definition}

By construction, one can see that for constants $\alpha, \beta$ and $\left( r,s \right) $ tensors $S,T$,
\begin{align}
    \mathcal{L}_X \left( \alpha S + \beta T \right) = \alpha \mathcal{L}_X S + \beta \mathcal{L}_X T
.\end{align}

To see how $\mathcal{L}_X$ acts in components, it is helpful to construct coordinates adapted to $X$. Near $p$ we can construct an $\left( n-1 \right) $-surface $\Sigma$ which is transverse to $X$, i.e. nowhere tangent. Pick coordinates $x^{i}$ on $\Sigma$ and assign the coordinate $\left( t,x^{i} \right) $ to the point a parameter distance $t$ along the integral curve of $X$ starting at $x^{i}$ on $\Sigma$.

In these coordinates. We can check that $X = \pdv{t}$ and $\phi_t^{X} \left( \tau, x^{i} \right) = \left( \tau + t, x^{i} \right)  $. So if $y^{\mu} = \left( \phi_t^{X} \right) ^{*}\left( x^{\mu} \right) $, then
\begin{align}
    \pdv{y^{\mu}}{x^{\nu}} = \delta^{\mu}_\nu
.\end{align}

Likewise,
\begin{align}
    \tensor{\left[ \left( \phi_t^{X} \right)^{*} T\right]}{^{\mu_1 \cdots \mu_r}_{\nu_1 \cdots \nu_s}}  \bigg|_{\left( t,x^{i} \right)} = \tensor{T}{^{\mu_1 \cdots \mu_r}_{\nu_1 \cdots \nu_s}} \bigg|_{\left( \tau + t, x^{i} \right) } 
.\end{align}

Thus, in this system of coordinates, the Lie derivative has components 
\begin{align}
    \tensor{\left( \mathcal{L}_X \right)}{^{\mu_1 \cdots \mu_r}_{\nu_1 \cdots \nu_s}} \bigg|_p = \pdv{\tensor{T}{^{\mu_1 \cdots \mu_r}_{\nu_1 \cdots \nu_s}}}{t} \bigg|_{p}
.\end{align}

So in these coordinates, $\mathcal{L}_X$ acts on components by $\pdv{t}$. In particular, we immediately see
\begin{align}
    \mathcal{L}_X \left( S \otimes T \right) = \left( \mathcal{L}_X S \right) \otimes T + S \otimes \mathcal{L}_X T
.\end{align}

Observe that $\mathcal{L}_X$ also commutes with contraction.

To write $\mathcal{L}_X$ in a coordinate free fashion, we can simply seek a basis independent expression that agrees with $\mathcal{L}_X$ in these coordinates.

\begin{example}
    For a function, $\mathcal{L}_X f = \pdv{f}{t} = X \left( f \right) $ in these coordinates.
\end{example}

For a vector field $Y$ we observe that as in these coordinates $X^{\sigma} = \left( 1,0,\cdots,0 \right) $,
\begin{align}
    \dv{Y^{\mu}}{t} &= X^{\sigma} \dv{x^{\sigma}} \left( Y^{\mu} \right)  \\
    \dv{Y^{\mu}}{t} &= X^{\sigma} \dv{x^{\sigma}} \left( Y^{\mu} \right) -  \underbrace{Y^{\sigma} \pdv{x^{\sigma}} X^{\mu}}_{0} \\
   \mathcal{L}_X Y &= \left[ X, Y \right]^{\mu}
.\end{align}

\begin{exercise}
    In any coordinate basis, show that if $\omega_a$ is a covector field,
    \begin{align}
        \left( \mathcal{L}_X \omega \right)_{\mu} = X^{\sigma} \partial_\sigma \omega_\mu + \omega_\sigma \partial_\mu X^{\sigma}
    .\end{align}

    If $\nabla$ is torsion free,
    \begin{align}
        \left( \mathcal{L}_X \omega \right)_a = X^{b} \nabla_b \omega_a + \omega_b \nabla_a X^{b}
    .\end{align}
    If $g_{ab}$ is a metric tensor and $\nabla$ the Levi Civita connection, then,
    \begin{align}
        \left( \mathcal{L}_X g\right)_{ab} = \nabla_a X_b + \nabla_b X_a 
    .\end{align}
\end{exercise}
See example sheet 2.9.

If $\phi_t$ is a one-parameter family of isometries for a manifold with metric $g$. Then $\mathcal{L}_X g = 0$.

Conversely, if $\mathcal{L}_X g = 0$, then $X$ generates a one parameter family of isometries.

\begin{definition}
    A vector field $K$ satisfying $\mathcal{L}_K g = 0$ is called a \textbf{Killing vector field}. It satisfies \textit{Killing's equation},
    \begin{align}
        \nabla_a K_b + \nabla_b K_a = 0
    ,\end{align}
    where $\nabla$ is the Levi-Civita connection.
\end{definition}

\begin{lemma}
    Suppose that $K$ is Killing and $\lambda : I \to M$ is a geodesic of the Levi Civita connection. Then $g_{ab} \dot{\lambda}^{a} K^{b}$ constant along $\lambda$.
\end{lemma}

\begin{proof}
    With
    \begin{align}
        \dv{t} \left( K_b \dot{\lambda}^{b} \right) &= \dot{\lambda^{a}} \nabla_a \left( K_b \dot{\lambda}^{b} \right) \\
        &= \left( \nabla_a K_b \right) \dot{\lambda}^{a} \dot{\lambda}^{b} + K_b \dot{\lambda}^{a} \nabla_a \dot{\lambda}^{b} \\
        &= \left( \nabla_{(a} K_{b)} \right) \dot{\lambda}^{a} \dot{\lambda}^{b} + K_b \left( \nabla_{X_\lambda} X_\lambda \right)^{b} \\
        &= 0 + 0
    ,\end{align}
    as $K$ is Killing and $\lambda$ is a geodesic respectively.
\end{proof}

Enumerating all possible solutions to Killing's equations, in $3+1$ Minkowski space we find $4$ translations, $3$ boosts and $3$ rotations.
