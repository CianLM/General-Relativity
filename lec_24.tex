\lecture{24}{04/12/2024}{Second fundamental form}

We deduce from the formulae for $\delta R_g$ and $\delta \dd{\text{vol}}_{g}$ that
\begin{align}
    \delta S_{\text{EH}} = \frac{1}{16\pi} \int_M \left( \left( \frac{1}{2} g^{ab} R_g - R^{ab}  \right) \delta g_{ab} + \nabla_c X^{c} \dd{\text{vol}}_g \right) 
.\end{align}

Notice that once one integrates the last term away by parts, this is exactly,
\begin{align}
    \delta S_{\text{EH}} = \frac{1}{16\pi} \int_M - G^{ab} \delta g_{ab} \dd{\text{vol}}_g
,\end{align}
where we've used the fact that $\delta g$ and hence $X$ vanish on $\partial M$ to drop the last term using the divergence theorem.

We immediately see that the variation $\delta S_\text{EH} = 0$ for all variations $\delta g_{ab}$ if and only if $g_{ab}$ solves the vacuum Einstein equations.

Suppose we also have a contribution from matter fields,
\begin{align}
    S_\text{tot} = S_{\text{EH}} + S_\text{matter}
,\end{align}
where
\begin{align}
    S_\text{matter} = \int_M L \left[ \Phi, g \right] \dd{\vol}_g
.\end{align}

Under a variation $g \to g + \delta g$, we must have that
\begin{align}
    \delta S_\text{matter} = \frac{1}{2} \int_M T^{ab} \delta g_{ab} \dd{\vol}_g
,\end{align}
for some symmetric 2-tensor $T^{ab}$. Varying $g$ in $S_\text{tot}$ gives
\begin{align}
    G^{ab} = 8 \pi T^{ab}
,\end{align}
namely, Einstein's equations.

\begin{example}
    If $\psi$ is a scalar field and $L_\text{matter} =-\frac{1}{2} g^{ab} \nabla_a \psi \nabla_b \psi$, then under $g \to g + \delta g$,
    \begin{align}
        \delta S_\text{matter} &= -\int_M \left[ \frac{1}{2} \delta g^{ab} \nabla_a \psi \nabla_b \psi \dd{\vol}_g + \frac{1}{2} g^{ab} \nabla_a \psi \nabla_b \psi \delta \left( \dd{\vol}_g \right) \right]  \\
        &= \frac{1}{2} \int_M \left( g^{ac} g^{bd} \nabla_c \psi \nabla_d \psi \delta g_{ab} -\frac{1}{2} g^{cd} \nabla_c \psi \nabla_d \psi g^{ab} \delta g_{ab}\right) \dd{\vol}_g  \\
        &= \frac{1}{2} \int_M T^{ab} \delta g_{ab} \dd{\vol}_g 
    ,\end{align}
    where we can read off
    \begin{align}
        T^{ab} = \nabla^{a} \psi \nabla^{b} - \frac{1}{2} g^{ab} \nabla_c \psi \nabla^{c} \psi
    .\end{align}
    \begin{exercise}
        Show that varying $\psi \to \psi + \delta \psi$ gives the wave equation, $\nabla_c \nabla^{c} = 0$.
    \end{exercise}
\end{example}

It can be shown that diffeomorphism invariance of the matter action implies $\nabla_a T^{ab} = 0$.

\subsection{The Cauchy Problem for Einstein's equations}

We expect Einstein's equations can be solved given data on a spacelike hypersurface, $\Sigma$. The natural question is \textit{what is the right data}?

Suppose that $\iota : \Sigma \to M$ is an embedding such that $\iota \left( \Sigma \right)$ is spacelike. Then $h = \iota^{*}\left( g \right) $ is Riemannian.

% fig

\begin{definition}
    Let $n$ be a choice of unit normal to $\iota \left( \Sigma \right) $. We define for $X,Y$ vector fields on $\Sigma$ the \textbf{second fundamental form} to be
    \begin{align}
        k \left( X,Y \right) = \iota^{*}\left( g \left( , \nabla_{\widetilde{X}} \widetilde{Y} \right)  \right) 
    ,\end{align}
    where we have the push forwards $\iota_* X = \widetilde{X}$ and $\iota_* Y = \widetilde{Y}$ on $\iota \left( \Sigma \right) $.
\end{definition}

We pick local coordinates $\{y^{i}\} $ on $\Sigma$ and $\{x^{\mu}\}$ on $M$ such that $\iota : \left( y^{1}, y^{2}, y^{3} \right) \mapsto \left( 0,y^{1}, y^{2}, y^{3} \right) $. Then $n_a \propto \dd{x}^{0}$ and say $n_\mu = \alpha \delta^{0}_\mu$. If $X,Y$ are vector fields on $\Sigma$, say $X = X^{i} \pdv{y^{i}}$, $Y = Y^{i} \pdv{y^{i}}$. Take
\begin{align}
    \widetilde{X} = X^{i} \pdv{x^{i}}, && \widetilde{Y} = Y^{i} \pdv{x^{i}}
.\end{align}

Then
\begin{align}
    k \left( X,Y \right) &= g_{\mu \nu} n^{\mu} \widetilde{X}^{\sigma} \nabla_\sigma \widetilde{Y}^{\nu} \\
    &= \alpha \delta^{0}_\nu \widetilde{X^{\sigma}} \left( \partial_\sigma \widetilde{Y}^{\nu} + \tensor{\Gamma}{_{\sigma}^{\nu}_\tau} \widetilde{Y}^{\tau} \right)  \\
    &= \alpha \tensor{\Gamma}{_{i}^{0}_{j}} X^{i} Y^{j}
,\end{align}
and therefore $k$ is a symmetric 2-tensor on $\Sigma$. 

We can show that if $\left( M,g \right) $ solves the vacuum Einstein equations, then the Einstein constraint equations hold such that
\begin{align}
    \nabla_{i}^{\left( h \right) } \tensor{k}{^{i}_j} - \nabla_j^{\left( h \right) } \tensor{k}{^{i}_i} = 0 && R_h - \tensor{k}{^{i}_j} \tensor{k}{^{j}_i} + \tensor{k}{^{i}_i} \tensor{k}{^{j}_j} = 0
.\end{align}

Conversely, if we are given
\begin{itemize}
    \item $\Sigma$, a 3-manifold,
    \item $h$, a Riemannian metric on $\Sigma$,
    \item $k$, a symmetric 2-tensor on $\Sigma$
,\end{itemize}
such that these equations hold, then there exists a solution $\left( M,g \right) $ of the vacuum Einstein equations, and an embedding $\iota : \Sigma \to M$ such that $h$ is the induced metric on $\Sigma$, and $k$ is the second fundamental form of $\iota \left( \Sigma \right) $.

We also have uniqueness of this solution on an open subset of $M$ (by Choque-Bruhat-Geroch).

