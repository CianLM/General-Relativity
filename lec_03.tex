\lecture{3}{16/10/2024}{Tangent Spaces}

\subsection{The Tangent Space is a Vector Space}


\begin{proposition}
    The set of tangent vectors to curves at $\vb{p}$ forms a vector space, $T_{\vb{p}} M$ of dimension $n = \dim M$. We call $T_{\vb{p}}M$, the \textbf{tangent space} to $M$ at $\vb{p}$.
\end{proposition}

\begin{proof}
    Given $X_{\vb{p}}, Y_{\vb{p}}$ are tangent vectors, we need to show that $\alpha X_{\vb{p}} + \beta Y_{\vb{p}}$ is a tangent vector for $\alpha, \beta \in \R$.
    Let $\lambda , \kappa$ be smooth curves with $\lambda \left( 0 \right) = \kappa \left( 0 \right) = \vb{p}$ and whose tangent vectors at $\vb{p}$ are $X_{\vb{p}}$ and $Y_{\vb{p}}$ respectively.
    Let $\left( \mathcal{O}, \phi \right) $ be a chart with $p \in \mathcal{O}$ such that $\phi \left( \vb{p} \right) = 0$. We call this a \textit{chart centered at $\vb{p}$}.

    Let $\nu \left( t \right) = \phi^{-1} \left[ \alpha \phi \left( \lambda \left( t \right)  \right) + \beta \phi \left( \kappa \left( t \right)  \right)  \right] $ where notice $\nu \left( 0 \right) = \phi^{-1} \left( 0 \right) = \vb{p}$.

    From \cref{eq:tangent_vector_2}, we have that if $Z_p$ is the tangent to $\nu$ at $\vb{p}$, we have
    \begin{align}
        Z_{\vb{p}}\left( f \right) &= \dv{t} \left( f \left( \nu \left( t \right)  \right)  \right) \bigg|_{0} \\
        &= \pdv{F}{x^{\mu}} \bigg|_{0} \dv{t} \left[ \alpha x^{\mu} \left( \lambda\left( t \right)\right)   + \beta x^{\mu}\left( \kappa \left( t \right)  \right)   \right] \bigg|_{t=0}\\
        &= \alpha \pdv{F}{x^{\mu}} \bigg|_{0} \dv{t} x^{\mu}\left( \lambda \left( t \right)  \right) \bigg|_{t=0} + \beta \pdv{F}{x^{\mu}} \bigg|_{0} \dv{t} x^{\mu}\left( \kappa \left( t \right)  \right) \bigg|_{t=0}  \\
        &= \alpha X_{\vb{p}}\left( f \right) + \beta X_{\vb{p}}\left( f \right) 
    ,\end{align}
    as desired. Therefore $T_{\vb{p}}M$ is a vector space.
\end{proof}

To see that $T_{\vb{p}}M$ is $n$-dimensional, consider the curves
\begin{align}
    \lambda_{\mu} \left( t \right) = \phi ^{-1} \left( 0, \cdots, 0, \underbrace{t}_{\mu\text{th component}},0,\cdots, 0 \right) 
.\end{align}

We denote the tangent vector to $\lambda_\mu$ at $\vb{p}$ by $\left( \pdv{x^{\mu}} \right)_{\vb{p}} $. 

\begin{note}
    This is \textbf{not} a differential operator.
\end{note}

However observe that by definition, we have
\begin{align}
    \left( \pdv{x^{\mu}} \right)_{\vb{p}}\left( f \right) = \pdv{F}{x^{\mu}} \bigg|_{\phi \left( \vb{p} \right) = 0}
,\end{align}
and thus it acts like a differential operator in $\R^{n}$ on the coordinates of the chart.

The vectors $\left( \pdv{x^{\mu}} \right)_{\vb{p}}$ are linearly independent. Otherwise $\exists  \alpha^{\mu} \in \R$ not all zero such that
\begin{align}
    \alpha^{\mu} \left( \pdv{x^{\mu}} \right)_{\vb{p}}  = 0
,\end{align}
which implies
\begin{align}
    \alpha^{\mu} \pdv{F}{x^{\mu}} = 0
,\end{align}
for all $F$. Setting $F = x^{\nu}$ gives $\alpha^{\nu} = 0$ and thus these vectors are linearly independent and spanning. They are therefore a basis for the vector space.

Further one can see that $\left( \pdv{x^{\mu}} \right)_{\vb{p}} $ form a basis for $T_{\vb{p}}M$, since if $\lambda$ is any curve with tangent $X_{\vb{p}}$ at $\vb{p}$, we have
\begin{align}
    X_{\vb{p}} \left( f \right) = \pdv{F}{x^{\mu}} \bigg|_{x=0} \dv{t} x^{\mu} \left( \lambda \left( t \right)  \right) = X^{\mu} \left( \pdv{x^{\mu}} \right)_{\vb{p}} \left( f \right) 
,\end{align}
where $X^{\mu} = \dv{t} x^{\mu} \left( \lambda \left( t \right)  \right) \bigg|_{t= 0}$ are the \textbf{components} of $X_{\vb{p}}$ with respect to the basis $\{\left( \pdv{x^{\mu}} \right)_{\vb{p}} \}_{\mu = 1,\cdots,n} $ for $T_{\vb{p}}M$.

\begin{note}
    The basis $\{\left( \pdv{x^{\mu}} \right)_{\vb{p}} \}_{\mu = 1,\cdots,n} $ depends on the coordinate chart $\phi$.
\end{note}

Suppose we choose another chart $\left( \mathcal{O}', \phi' \right) $, again centered at $\vb{p}$. We write $\phi' = \left(\left( x^{'} \right)^{1}, \cdots, \left( x' \right)^{n}   \right) $. Then if $F' = f \circ \phi'^{-1}$, we have
\begin{align}
    F \left( x \right) &= f \circ \phi^{-1}\left( x \right) \\
    &= f \circ \phi'^{-1} \circ \phi'\circ \phi^{-1} \left( x \right) \\
    &= F' \left( x' \left( x \right)  \right)
.\end{align}

Therefore,
\begin{align}
    \left( \pdv{x^{\mu}} \right)_{\vb{p}} \left( f \right) &=  \pdv{F}{x^{\mu}} \bigg|_{\phi \left( \vb{p} \right) }\\
    &= \left( \pdv{x'^{\nu}}{x^{\mu}} \right) \bigg|_{\phi \left( \vb{p} \right) }  \left( \pdv{F'}{x'^{\nu}} \right) \bigg|_{\phi' \left( \vb{p} \right) } \\
    &= \left( \pdv{x'^{\nu}}{x^{\mu}} \right)  \bigg|_{\phi \left( \vb{p} \right) } \left( \pdv{x'^{\nu}} \right)_{\vb{p}} \left( f \right)
.\end{align}

We then deduce that
\begin{align}
    \left( \pdv{x^{\mu}} \right)_{\vb{p}} = \left( \pdv{x'^{\nu}}{x^{\mu}} \right) \bigg|_{\phi \left( \vb{p} \right) } \left( \pdv{x'^{\nu}} \right)_{\vb{p}} 
.\end{align}

Let $X^{\mu}$ be components of $X_{\vb{p}}$ with respect to the basis $\left( \pdv{x^{\mu}} \right)_{\vb{p}}$, and $X'^{\mu}$ be components of $X_{\vb{p}}$ with respect to the basis $\left( \pdv{x^{\mu}} \right)_{\vb{p}} $ such that
\begin{align}
    X_{\vb{p}} &= X^{\mu} \left( \pdv{x^{\mu}} \right)_{\vb{p}}  = X'^{\mu} \left( \pdv{x'^{\mu}} \right)_{\vb{p}} \\
    &= X^{\mu} \left( \pdv{x'^{\sigma}}{x^{\mu}} \right) \left( \pdv{x'^{\sigma}} \right)_{\vb{p}}
,\end{align}
and therefore
\begin{align}
    X'^{\mu} &= \left( \pdv{x'^{\mu}}{x^{\nu}} \right) X^{\nu}
.\end{align}

\begin{note}
    We do note have to choose a coordinate basis such as $\left( \pdv{x^{\mu}} \right)_{\vb{p}}$. With respect to a general basis $\{e_{\mu}\} $, for $T_{\vb{p}}M$, we can write $X_{\vb{p}}= X^{\mu} e_{\mu}$ for $X^{\mu} \in \R$.

    We always use summation convention, contracting covariant indices with contravariant indices.
\end{note}

\subsection{Covectors}

Recall that if $V$ is a vector space over $\R$, the dual space $V^{*}$ is the space of linear maps $\phi : V \to \R$. If $V$ is $n$-dimensional then so is $V^{*}$ (the spaces are then isomorphic). Given a basis $\{e_{\mu}\} $ for $V$, we can define the dual basis $\{f^{\mu}\} $ for $V^{*}$ by requiring that
\begin{align}
    f^{\mu} \left( e_{\nu} \right)  = \tensor{\delta}{^{\mu}_\nu} = \begin{cases}
        1, & \text{~if $\mu = \nu$,}\\
        0, & \text{~if $\mu \neq \nu$.}
    \end{cases}
\end{align}

If $V$ is finite dimensional, then $V^{**}=\left( V^{*} \right)^{*} \simeq  V$. Namely, to an element $X \in V$, we assign the linear map 
\begin{align}
    \Lambda_X : V^{*} \to \R, \\
    \Lambda_X \left( \omega \right) = \omega \left( X \right) 
,\end{align}
for $\omega \in V^{*}$.

\begin{definition}
    The dual space of $T_{\vb{p}}M$ is denoted $T^{*}_{\vb{p}}M$ and is called the \textbf{cotangent space} to $M$ at $\vb{p}$. An element of this space is a \textbf{covector} at $\vb{p}$. If $\{e_{\mu}\} $ is a basis for $T_{\vb{p}}M$ and $\{f^{\mu}\} $ is the dual basis for $T^{*}_{\vb{p}}M$, we can expand a covector $\eta$ as
    \begin{align}
        \eta = \eta_{\mu} f^{\mu}
    ,\end{align}
    for \textbf{components} $\eta_\mu \in \R$.
\end{definition}

